\documentclass{psu-report}
\usepackage{psu-thesis-chicago}
\usepackage{todonotes}
\usepackage{amsmath,amssymb,amsfonts,amsthm,mathtools}
\usepackage{siunitx}
\usepackage{tikz,pgfplots,pgfplotstable}
\pgfplotsset{compat=newest}

\addbibresource{refs.bib}

\title{Organ Pipe Tuning}
\author{
    Meghana Cyanam \\
    Hai Nguyen \\
    Gabriel Pinochet-Soto \\
}
\doctype{report}
\psumonth{April}
\psuyear{2025}

\professorOne{Daniel Taylor-Rodriguez}
\professorTwo{Jacob Schultz}
\clientOne{Arno Patin}

\documentformat{monograph}


\begin{document}

\maketitle
\makecopyright

\begin{psuabstract}
    Intonation, or voicing, in pipe organ technology involves adjusting each
    pipe to produce its optimal sound.
    This process entails determining the appropriate tone that harmonizes with
    the organ’s style, the architectural setting, and the preferences of its
    owner.
    While the “perfect” tone is largely subjective, certain aspects, such as 
    volume and harmony, can be quantified using contemporary tools.
    
    The relationship between the diameter of the pipe’s toe, the size of the toe
    hole, airflow, and frequency is paramount in comprehending and optimizing
    organ pipe performance.
    These parameters interact to influence the pipe’s tonal characteristics,
    acoustic intensity, and harmonic structure. Adjustments in the toe hole size
    regulate airflow, affecting both frequency and tonal balance, while the pipe
    toe diameter significantly shapes the overall sound profile.
    Exploring these connections elucidates the intricacies of voicing and 
    scaling processes, providing insights into both traditional methods and
    contemporary approaches employing acoustic measurements.

    Our goal is to clarify the relationship between the Ising number and the
    parameters of the organ pipe, and to provide a model that can be used to
    dimension the pipes before the voicing stage.

    % Prior to installation in the organ, the pipes undergo a meticulous
    % fine-tuning process on a specialized bench.
    % This occurs after the design of the pipes, encompassing their size, shape,
    % and materials, has been finalized. The process of designing pipes, known as
    % scaling, has a history dating back to the Middle Ages and adheres to
    % well-established principles of the craft.
    % Despite the availability of advanced tools, intonation remains a challenging
    % aspect. Experts have dedicated centuries to studying and refining intonation
    % techniques, employing diverse approaches.
    % One notable method involves a mathematical formula developed by Hardmuth
    % Ising, which has been extensively studied and discussed by other
    % researchers.
\end{psuabstract}

% \begin{psudedication}
% \end{psudedication}
% \begin{psuacknowledgments}
% \end{psuacknowledgments}

\tableofcontents
\clearpage

% \listoftables
% \clearpage

% \listoffigures
% \clearpage

% \begin{psuglossary}
% \begin{description}
%     \item[LaTeX] A document preparation system
% \end{description}
% \end{psuglossary}

% \begin{psupreface}
% \end{psupreface}

\startbody

\chapter{Introduction}

\todo{Add introduction}

\section{Background}

\todo{Add background}

\chapter{Action Plan}

In this Chapter, we discuss different approaches for successfully improving the
organ pipe tuning process.
We discuss the project goals and objectives, potential solutions, and the
expected outcomes of the project.
Subsequently, we present the data available, some challenges and concerns that
we encountered.
Proposed methodologies are discussed and briefly explained.
Finally, we present the project timeline and the expected deliverables.

\section{Project Goals and Objectives}

The main goal of this project concerns the improvement of the organ pipe tuning
process.
As the organ pipe design is a complex process, and usually is not a reversible
process, it is important to ensure that the design of the pipes is done
correctly; this presents a key constraint for the tuning process.
The parameters that are involved are diverse, and despite being computable or
measurable, the access to this information does not simplify the tuning process.

A particular subgoal of this project is to improve the Ising
formula~\autocite{1971Ising-1}, which reads
\begin{equation}
    \label{eq:ising}
    \mathsf{I}
    =
    \frac{v}{\omega}\sqrt{\frac{d}{h^3}}
    =
    \frac{1}{\omega}\sqrt{\frac{2 P d}{\rho h^3}},
\end{equation}
where \(\mathsf{I}\) is the Ising number, \(\omega\) is the (target) frequency
of the pipe, \(v\) is the jet initial velocity, \(d\) is the jet initial
thickness, \(h\) is the cut-up height (or the length of the mouth), \(P\) is the
blowing pressure, and \(\rho\) is the density of the air.
From \autocite{1971Ising-1, 2025Lilj-1}, it is suggested that the Ising number
should be constant, preferably close to \( \mathsf{I} \approx 2\).

We would like to include the flow rate in the Ising formula~\ref{eq:ising}, so
this methodology can be used to dimension the pipes in a reliable way, before
the voicing stage.

\subsection{Key questions}

The following questions are key to the project, and we aim to address them to
our best ability:
\begin{enumerate}
    \item Can the Ising formula~\ref{eq:ising} be improved, refined, or
        corrected, in order to match measured data?
    \item Can the flow rate be included in the Ising formula~\ref{eq:ising}?
    \item Can we obtain a model reliable enough, that can be used to dimension the
        pipes before the voicing stage?
\end{enumerate}

\section{Available data sources}

The main source of data in provided by the client.
\todo{Add more details about the data sources}

\section{Methodology}

We intend to use a combination of semianalitical methods, numerical methods,
machine learning techniques, and data analysis techniques to address the
questions presented in the previous section.

\subsection{Semianalytical methods}

As described in \autocite{2004RosFle-1}, the physics of the organ pipes relies
on the wave equation.
Exact solutions for the time-harmonic, longitudinal, wave equation can be
derived using the method of separation of variables.

These solutions give the eigenvalues (i.e., the frequencies). 
The study of these frequencies -- in particular, the fundamental
frequency -- is important for the tuning process.

We intendo to explore basic semianalytical methods, such as the method of
separation of variables, to study the eigenvalues of the wave equation.
We rely on available literature~\autocite{2004RosFle-1, 2012RosFle-1}.

% \todo{Add more details about the semianalytical methods}

\subsection{Numerical methods: Finite Element Method}

The Finite Element Method (FEM) is a numerical method for solving partial
differential equations (PDEs) and integral equations.
As the govening equations of the organ pipe are PDEs, we can use FEM to solve
them.
Standars techniques are available, and we will not delve into the details of the
FEM.
See~\autocite{2021ErnGue-1, 2021ErnGue-2, 2019VazKeeDem-1} for a
comprehensive introduction to the FEM and PML methods.

\subsection{Machine learning techniques}

Modern machine learning techniques can be used to model the data, and to
predict the Ising number.
We use a simple neural network to model the data, and to predict the Ising
number.
\todo{Add more details about the machine learning techniques}
The final layer consists of a split layer, where one components relative to the
Ising number uses a Softplus activation function \todo{Ref needed}, and the
partials ratios are given by sigmoid activation functions \todo{Ref needed}.

\subsection{Generalized Additive Models}

\todo{Add more details about the Generalized Additive Models}

\section{Project timeline}

\todo{Description}
The project timeline is the following:
\begin{itemize}
    \item Weeks 1-2: Data adquisition, cleaning, and preliminary analysis.
    \item Weeks 3-4: Preliminary exploration of the data.
    \item Weeks 5-8: Implementation of the methodologies.
    \item Weeks 9: Synthesis of the results.
    \item Weeks 10: Final report and presentation.
\end{itemize}

\nocite{*} % This will include all references in the bibliography
\printchicagobibliography

% \chapter*{Notes}
% \addcontentsline{toc}{chapter}{Notes}
% \printchicagonotes

% \appendix

\end{document}
